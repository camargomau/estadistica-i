\documentclass[12pt, a4paper]{article}
\usepackage[margin=1in]{geometry}
\usepackage{utilities/preamble}
\makeindex

\newcommand{\titulo}{Historia de la Estadística}
\newcommand{\fecha}{28 de agosto de 2023}

\begin{document}
\sffamily
\pagenumbering{gobble}
\begin{titlepage}
    \begin{center}
        \includegraphics[width=0.15\textwidth]{assets/unam.png}
        \hspace{0.6\textwidth}
        \includegraphics[width=0.15\textwidth]{assets/fes.png}

        \vspace*{5cm}
        \LARGE
        \textbf{\titulo}

        \vspace{1cm}
        \large
        Camargo Badillo Luis Mauricio \\
        \vspace{0.5cm}
        \textit{\fecha}

        \vfill

        \vspace{0.5cm}
        Estadística I \\
        \textbf{Matemáticas Aplicadas y Computación}\\
    \end{center}
\end{titlepage}


Actualmente, vivimos en una era llena de información y datos, por lo que es imposible imaginar una realidad en la que la estadística no exista. No obstante, el objeto y razón de ser de la estadística no siempre fueran tan claros y evidentes como pensaríamos hoy en día.

El propósito de esta investigación es explorar los momentos clave de la historia de la estadística, desde sus inicios tempranos hasta la actualidad. Se presentará un panorama amplio del desarrollo de la estadística que permitirá obtener un mejor entendimiento de no solo su importancia en el mundo actual, sino también de su objeto de estudio y sus motivaciones.

Podemos obtener un pequeño vistazo a la historia de la estadística si analizamos la etimología de la palabra inglesa \textit{statistics}:

\begin{quote}
	[Atestiguada con este uso desde] 1770, 'la ciencia que se encarga de los datos sobre la condición de un Estado o comunidad', del alemán \textit{Statistik}, popularizado y tal vez acuñado por el científico político alemán Gottfried Achenwall […], del latín moderno \textit{statisticum (collegium)} '(curso sobre) asuntos de Estado', del italiano \textit{statista} 'experto en asuntos de Estado', del latín \textit{status} 'estación, posición, lugar; orden, disposición, condición', figurativamente 'orden público, organización comunitaria' […]. (Online Etymology Dictionary, s.f.)
\end{quote}

Cabe aclarar que según la Real Academia Española (s.f.), la etimología de la palabra castellana \textit{estadística} es prácticamente la misma.

Analizando esta etimología, nos damos cuenta de que, a pesar de que hoy en día pocas veces pensamos en la palabra \textit{estadística} como relacionada a \textit{Estado}, estas dos palabras son en realidad cognados, palabras que tienen un mismo origen. Esto nos dice que la estadística muy probablemente tuvo fuertes raíces en el análisis de datos de naturaleza poblacional, como los que buscaría analizar un Estado o una comunidad. Ahondaré en esta vertiente a lo largo de esta investigación para comprobar que efectivamente fue así.

\section{Civilizaciones Antiguas}

Podemos rastrear el inicio de los métodos estadísticos hasta las civilizaciones antiguas. Ya desde el año 4000 a.C., los babilonios llevaban a cabo censos con el objetivo de calcular cuánta comida se necesitaba para alimentar a cada miembro de la población.

Otras civilizaciones antiguas que tuvieron interés en datos estadísticos fueron Egipto y China, quienes también se apoyaron de censos para propósitos laborales, territoriales o de riqueza. En particular, el censo tomado por la dinastía Han en el año 2 d.C. es considerado por expertos como relativamente preciso. Este censo registró que en ese entonces, en la China antigua vivían 57.67 millones de personas.

Los romanos, una de las civilaciones antiguas más importantes, utilizaron los censos para tener bajo control su cada vez más creciente territorio y población, incluso comenzando a utilizar los datos obtenidos para determinar los impuestos que se le cobrarían a sus ciudadanos.

\section{}

\section{Conclusiones}

\section{Introducción}

\end{document}
