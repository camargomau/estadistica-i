\documentclass[12pt, a4paper]{article}
\usepackage[margin=1in]{geometry}
\usepackage{utilities/preamble}

\newcommand{\titulo}{Historia de la Estadística}
\newcommand{\fecha}{28 de agosto de 2023}

\begin{document}
\sffamily
\begin{titlepage}
    \begin{center}
        \includegraphics[width=0.15\textwidth]{assets/unam.png}
        \hspace{0.6\textwidth}
        \includegraphics[width=0.15\textwidth]{assets/fes.png}

        \vspace*{5cm}
        \LARGE
        \textbf{\titulo}

        \vspace{1cm}
        \large
        Camargo Badillo Luis Mauricio \\
        \vspace{0.5cm}
        \textit{\fecha}

        \vfill

        \vspace{0.5cm}
        Estadística I \\
        \textbf{Matemáticas Aplicadas y Computación}\\
    \end{center}
\end{titlepage}


\tableofcontents
\newpage

La estadística es la ciencia que se encarga de recopilar, analizar, resumir e interpretar datos para obtne conclusiones sobre una o más poblaciones.

Actualmente, vivimos en una era llena de información y datos, por lo que nos es casi imposible imaginar una realidad en la que la estadística no exista. No obstante, el objeto y razón de ser de la estadística no siempre fueran tan claros y evidentes como pensaríamos hoy en día.

El propósito de esta investigación es explorar los momentos clave de la historia de la estadística, desde sus inicios tempranos hasta la actualidad. Se presentará un panorama amplio del desarrollo de la estadística que permitirá obtener un mejor entendimiento de no solo su importancia en el mundo actual, sino también de su objeto de estudio y sus motivaciones.

Podemos obtener un pequeño vistazo a la historia de la estadística si analizamos la etimología de la palabra inglesa \textit{statistics}:

\begin{quote}
	[Atestiguada con este uso desde] 1770, 'la ciencia que se encarga de los datos sobre la condición de un Estado o comunidad', del alemán \textit{Statistik}, popularizado y tal vez acuñado por el científico político alemán Gottfried Achenwall […], del latín moderno \textit{statisticum (collegium)} '(curso sobre) asuntos de Estado', del italiano \textit{statista} 'experto en asuntos de Estado', del latín \textit{status} 'estación, posición, lugar; orden, disposición, condición', figurativamente 'orden público, organización comunitaria' […]. (Online Etymology Dictionary, s.f.)
\end{quote}

Cabe aclarar que según la Real Academia Española (s.f.), la etimología de la palabra castellana \textit{estadística} es prácticamente la misma.

Analizando esta etimología, nos damos cuenta de que, a pesar de que hoy en día pocas veces pensamos en la palabra \textit{estadística} como relacionada a \textit{Estado}, estas dos palabras son en realidad cognados, palabras que tienen un mismo origen. Esto nos dice que la estadística muy probablemente tuvo fuertes raíces en el análisis de datos de naturaleza poblacional, como los que buscaría analizar un Estado o una comunidad. Ahondaré en esta vertiente a lo largo de esta investigación para comprobar que efectivamente fue así.

\section{Antiguas Civilizaciones}

Podemos rastrear el inicio de los métodos estadísticos hasta las civilizaciones antiguas. Ya desde el año 4000 a.C., los babilonios llevaban a cabo censos con el objetivo de calcular cuánta comida se necesitaba para alimentar a cada miembro de la población.

Otras civilizaciones antiguas que tuvieron interés en datos estadísticos fueron Egipto y China, quienes también se apoyaron de censos para propósitos laborales, territoriales o de riqueza. En particular, el censo tomado por la dinastía Han en el año 2 d.C. es considerado por expertos como relativamente preciso. Este censo registró que en ese entonces, en la China antigua vivían 57.67 millones de personas.

Los romanos, una de las civilaciones antiguas más importantes, utilizaron los censos para tener bajo control su cada vez más creciente territorio y población, incluso comenzando a utilizar los datos obtenidos para determinar los impuestos que se le cobrarían a sus ciudadanos.

\section{Primeros Pasos}

John Graunt, comerciante en paños inglés nacido en 1620 y muerto en 1674, es considerado como uno de los padres fundadores de la estadística y de la demografía, el estudio estadístico de las poblaciones humanas.

Sin duda alguna, la contribución más importante de Graunt fue la publicación de \textit{Observaciones naturales y políticas hechas a partir de los boletines de mortalidad} en 1662. Es aquí donde Graunt lleva a cabo lo que se considera como uno de los primeros trabajos de análisis demográfico y de estadística descriptiva que inspiró e impulsó el desarrollo de la disciplina en años posteriores.

En este documento, Graunt se basa en los datos recopilados por boletines del Londres de ese entonces, que incluían datos como el número de bautizados, el número de fallecidos y sus causas, etc. Graunt utilizó estos datos no solo para hacer análisis de mortalidad y estimados poblacionales, sino también para	sacar conclusiones respecto a la salud pública y la planeación urbana de esa época. El trabajo de Graunt plantó los cimientos para el pensamiento estadístico, introduciendo conceptos novedosos como las medias.

Es en esta fase de desarrollo de la estadística en la que William Petty, polímata inglés nacido en 1623 y muerto en 1687, también es relevante. Petty, en parte inspirado por el trabajo de Graunt, fue quien acuñó el término \textit{aritmética política} para hacer referencia al análisis cuantitativo de fenómenos sociales y económicos; ese es el término en inglés (\textit{political arithmetic}) que se utilizaba para referirse a lo que en alemán se llamaba \textit{Statistik}.

Petty también desarrolló diversas tecnicas para presentar datos estadísticos de forma visual, entre ellas las gráficas de barras y diagramas. Incluso llevó a cabo trabajos de análisis económicos y territoriales que tuvieron importantes implicaciones en Irlanda, donde condujo encuestas demográficas.

\section{Fortalecimiento}

Para el siglo XVIII, la estadística seguía siendo una práctica hasta cierto punto poco científica. A lo largo de este siglo, los practicantes de la estadística (recordemos, en ese entonces \textit{política aritmética}) se basaban en principalmente en conjeturas.

Es en esta fase donde la estadística y la probabilidad comenzaro na ir de la mano. Como no existían muchos métodos estadísticos solidos y confiables, una buena parte de la estadística de ese entonces se basaba en estimaciones obtenidas a partir de la probabilidad.

Por ejemplo, Pierre Simon de Laplace, polímata francés nacido en 1749 y muerto en 1827, utilizó la probabilidad para conjeturar la exactitud de las cifras de población que había determinado a partir de muestras poblacionales. Recordemos que Laplace es una de las figuras más importantes de la teoría de la probabilidad, haciendo importantes aportaciones como el teorema central del límite, que establecía que cualquier media tendrá apróximademente una distribución normal, dado un número suficiente de observaciones (Gómez Villegas, 2004).

Con el tiempo y con el cambio del siglo, poco a poco durante el siglo XIX tales métodos fuero ncayendo en desuso. La principal causa fue que los censos sistemáticos y regulares comenzaron a realizarse. Precisamente se considera como uno de los primeros el censo de los Estados Unidos que comenzó a realizarse cada 10 años a partir de 1790, pero otras naciones, principalmente europeas, siguieron esta práctica para mediados del siglo.

Es esta motivación por estudiar a las poblaciones humanas la que impulsa el desarrollo de la estadística, tal como la conocemos hoy en día. En 1832 surge una de las primeras organizaciones estadísticas, la sección F de la Asociación Británica para el Avance de la Ciencia. Es en este periodo cuando Adolphe Quetelet, sociólogo y matemático belga nacido en 1796 y muerto en 1874, comienza a impulsar los primeros intentos por utilizar matemáticas propiamente dichas para las ciencias sociales. Quetelet incluso buscó crear una nueva ciencia, que llamaría \textit{física social}; muchas veces buscó relacionar las matemáticas de las ciencias naturales con las ciencias sociales.

\section{Expansión a otras disciplinas}

Es igual durante este siglo cuando la estadística, poco a poco y sigilosamente, comienza a filtrarse hacia otras disciplinas. El mejor ejemplo de esta expansión es el método de los mínimos cuadrados, publicado por el renombrado matemático alemán Carl Friedierich Gauss en 1809. Este método fue utilizado para ampliamente en la astronomía, geodesia e incluso la psicología cuantitativa, sin olvidar que fue una pieza clave de la definición del metro, la unidad de medida.

Karl Pearson, matemático inglés nacido en 1857 y muerto en 1936, juega un papel importante en esta expansión de la estadística, pues entre 1891 y 1892 impartió conferencias donde introducía herramientas importantes como los estigmogramas, entigramas, histogramas, cartogramas y estereogramas. No solo eso, sino que también entre 1893 y 1906 publicó 100 artículos sobre la estadística y confudó \textit{Biometrika} en 1901, una revista que publicó trabajos sobre herencia y más posteriormente sobre la estadística aplicada a la biología (Gómez Villegas, 2004).

A principios del siglo XX es cuando surgen campos nuevos que no podrían existir sin la estadística, como la econometría, enfocada en series temporales económicas, y la psicometría, dedicada a las pruebas mentales.

Finalmente en este siglo la estadística logró solidificarse y establecerse como una verdadera ciencia y disciplina con sus propios méritos. Ronald A. Fisher, polímata británico nacido en 1890 y muerto en 1962, logra posicionarse como la figura más importante de la estadística de este periodo. En 1922 publicó \textit{La Fundamentación Matemática de la Estadística Teórica}, donde introdujo nociones muy importantes, como la de lmodelo estadístico y los conceptos de consistencia, eficiencia, precisión, validación, verosimilitud e información (Gómez Villegas, 2004).

\section{La estadística actual}

Durante el siglo XX, la estadística siguió posicionándose como una herramienta vital en el estudio de los datos de muchas disciplinas. Comenzó incluso a utilizarse en disciplinas que antes parecían tan alejadas, como la psicología, la ecología y la investigación terapéutica, por nombrar solo algunos ejemplos.  Hoy en día, la estadística es utilizada en casi todas las áreas de la vida, desde la agricultura y los negocios, hasta las leyes, la ingeniería y, claro, las ciencias sociales.

El surgimiento y explosión de la era computacional también logró impulsar la estadística a lugares antes impensables, pues las limitaciones humanas dejaron de ser una preocupación y los cálculos que antes eran considerados difíciles pasaron a ser sumamente sencillos. Además, a partir de 1930 se vio el surgimiento de la profesión de \textit{estadista} propiamente dicha, una profesión que se ha vuelto importante gracias al gran alcance de la estadística.

\section{Conclusiones}

La historia de la estadística es una de las historias más curiosas de surgimiento de una disciplina. Comenzó como una serie de métodos poco científicos con escasa sustentación, y mediante la misma necesidad del mundo y su creciente organización, orgánicamente logró convertirse en una de las disciplinas más importantes para el mundo actual, logrando permear muchas otras disciplinas. Sin duda alguna, la vida actual no existiría sin ella.

Debemos agradecer el trabajo de todas las personas que hicieron posible el desarrollo de esta disciplina, entre las que destacan John Graunt, Adolphe Quetelet, Karl Pearson y Ronald A. Fisher, como vimos a lo largo de este documento.

Es cierto, la estadística cumplió su aspiración de convertirse en «la lógica de la incertidumbre». Nunca fue muy ambiciosa, pero de alguna forma logró llegar a alturas nunca antes vistas para una disciplina que anteriormente ni siquiera era vista como propia.

\newpage

\section{Referencias Bibliográficas}

\begin{itemize}
	\item Gómez Villegas, M. A. (2004). Estadísticos Significativos. \textit{Historia de la Probabilidad y la Estadística}, 207-217.

	\item Kafadar, K. (2020, abril 24). Reinforcing the Impact of Statistics in Society. \textit{Journal of the American Statistical Association}. 491-500. \url{https://doi.org/10.1080/01621459.2020.1761217}.

	\item Michigan Tech. (2021, mayo 20). \textit{Every Number Counts: The Importance of Applied Statistics in Our Daily Lives}. \url{https://onlinedegrees.mtu.edu/news/every-number-counts-importance-applied-statistics-our-daily-lives-infographic}

	\item Online Etymology Dictionary. (s.f.). \textit{Statistics (n.)}. \url{https://www.etymonline.com/search?q=statistics}

	\item Office for National Statistics. (2016, enero 18). \textit{Census-taking in the ancient world}. \url{https://www.ons.gov.uk/census/2011census/howourcensusworks/aboutcensuses/censushistory/censustakingintheancientworld}

	\item Porter, T. M. (2023, julio 27). \textit{Probability and Statistics}. Encyclopedia Britannica. \url{https://www.britannica.com/science/probability}

	\item Real Academia Española. (s.f.). \textit{Estadístico, ca}. \url{https://dle.rae.es/estad%C3%ADstico}

	\item The Editors of Encyclopaedia Britannica. (2023, abril 20). \textit{John Graunt}. Encyclopedia Britannica. \url{https://www.britannica.com/biography/John-Graunt}

	\item Williams, T. A., Sweeney, D. J. \& Anderson, D. R. (2023, junio 20). \textit{Statistics}. Encyclopedia Britannica. \url{https://www.britannica.com/science/statistics}
\end{itemize}

\end{document}
